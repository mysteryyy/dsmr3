%& -job-name=presentation
\documentclass{beamer}
\usepackage{graphicx}
\usepackage{booktabs}
\usepackage[backend=biber]{biblatex}
\usepackage{color, colortbl}
\usepackage[most]{tcolorbox}
\usepackage{amsmath}
\addbibresource{sample.bib}
%Information to be included in the title page:
\title{Data Science Research Methods Assignment-3}
\usetheme{Copenhagen}
\useinnertheme{circles}
%\usecolortheme{albatross}
%\useoutertheme{Albatross}
%\usetheme[hideothersubsections]{/home/sahil/beamerports/PraterStreet}
\author{243655}
\institute{University of Sussex}
\date{2021}

\begin{document}

\frame{\titlepage}

\begin{frame}{Introduction}

The objective of the report was to analyze a movie dataset from imdb to make reccomendations on the type of movie the studio should consider making.Our budget for production was 1.5 million.
%\begin{figure}
%	\includegraphics[width=\textwidth,height=\textheight,keepaspectratio]{privpublic.jpeg}
%	\caption{Comparision between the two types} \nocite{luxtag}
%  \label{fig:boat1}
%\end{figure}

\end{frame}
\begin{frame}{Data Analyses}
%\begin{itemize}
%\item Capital Markets
\begin{itemize}
    \item First step was to control for budget and only consider movies with a budget of less than 1.5 million.
    \item Next, a new feature, 'profit\_percentage' was calculated from the data such that,\\
    \begin{tcolorbox}[colback=red!5,colframe=green!75!black,title=]
	    $$\text{profit\_percentage}=\left(\frac{\text{gross}}{\text{budget}}-1\right)\times100$$
	    \tcblower
	    where 'gross' is just the total earning of the movie\\
	    and 'budget' is the budget of the movie

\end{tcolorbox} 

    \item We try to analyze the gross and profit of different genres through a boxplot.
	\\[1ex]
\end{itemize}

%\item Asset Management
%\begin{itemize}
%    \item Fund Launch
%    \item Fund administration
%    \item Transfer agency in asset management
%	\\[1ex]
%\end{itemize}
%\item Banking and Lending
%\begin{itemize}
%    \item Credit Prediction and Credit Scoring
%    \item Asset Collateralization
%\end{itemize}
%
%
%
%
%\end{itemize}


\end{frame}
\begin{frame}
\begin{figure}
	\includegraphics[width=\textwidth,height=\textheight,keepaspectratio]{test100.jpg}
	\caption{Boxplot of Gross grouped by Genres} 
	%\nocite{luxtag}
  \label{fig:boat1}
\end{figure}

\end{frame}
\begin{frame}
\begin{figure}
	\includegraphics[width=\textwidth,height=\textheight,keepaspectratio]{test200.jpg}
	\caption{Boxplot of profit\_percentage grouped by Genres} 
	%\nocite{luxtag}
  \label{fig:boat1}
\end{figure}

\end{frame}
\begin{frame}
\begin{itemize}
\item Tried to analyze most prevelant subplots in these movies.\\
\end{itemize}

\vspace{10mm} %5mm vertical space
\centering
\begin{tabular}{lr}
\toprule
    plot\_keywords &  frequency of occurence \\
\midrule
           friend &                      16 \\
             love &                       9 \\
 independent film &                       8 \\
            drugs &                       7 \\
       friendship &                       7 \\
\bottomrule
\end{tabular}

\end{frame}

\begin{frame}{Hypothesis Generated from Data}

\begin{itemize}
\item On analysing the subplots, one sees a clear pattern that certain movie genres, for instance, Comedy/Drama/Music  have higher earnings.
\item With respect to profit\_percentage however,one sees that genres like Horror and Horror/Thriller have higher return on investment.
\item We try to test if the above mentioned genres have higher returns on average compared to other genres. 

\end{itemize}
%	\begin{itemize}
%
%\item Corda helps businesses in Banking, Capital Markets, Trade Finance, Insurance and beyond to transact directly and in strict privacy using smart contracts, reducing transaction and record-keeping costs and streamlining business operations.
%	\\[3ex]
%\item It is built for highly regulated industries.
%	\\[3ex]
%\item Delivers the core attributes of open source with enterprise functionality, services and support.
%\end{itemize}
%\end{frame}
%\begin{frame}{Challenges Faced by Corda}
%\begin{itemize}
%   \item  The absence of blocks makes the simplification of validation impossible and requires to perform complete verification of transaction history which will become more resource-consuming over time. 
%	    \\[1ex]
%    \item Corda is vulnerable to double-spending attacks which requires using special oracles to prevent attackers from performing such kind of attack.
%
%\end{itemize}
\end{frame}
\begin{frame}
\begin{tcolorbox}[colback=purple!5,colframe=blue!75!black,title=Example of a Hypothesis]
	\begin{math}
	 \mu_1  = \text{Mean of profit\_percentage of genre 'Horror'} \\
	 \mu_2  = \text{Mean of profit\_percentage of all genres \textbf{except Horror}}\\
		H_0   =  \mu_1\leq\mu_2 , \text{Null Hypothesis} \\ 
		H_1   =  \mu_1>\mu_2 , \text{Alternate Hypothesis} \\
	\end{math}
	 \tcblower
	 The above hypothesis is tested using a right-sided t-test as the 
	 sample size might be too small for a z-test.\\
	 If the p-value from the above test
	 is less than 0.05(our confidence value), we conclude that Horror movies indeed
	 earn higher than average compared to other genres.Aspect Ratios are tested in a similar
	 way.

\end{tcolorbox} 


%    The immutability and transparency of a Corda blockchain help empower financial institutions for gaining fast and secure access to up-to-date customer data. 
%		\nocite{alehub.io_2018},\nocite{corda}     
%\begin{itemize}
% \item Transaction finality.
% \item Ability to scale.
% \item Privacy.
% \item Legally identified parties.
% \item Developer productivity and enterprise integration.
%\end{itemize}

\end{frame}
\begin{frame}

\begin{tcolorbox}[colback=purple!5,colframe=blue!75!black,title=Bootstrapped Hypothesis Testing]
Another type of hypothesis test was performed using bootstrapping to confirm the results obtained from t-test.In this version, bootstrapped sample of both the sets that are to be compared were taken and then the mean was calculated for each sample.These samples of means were then compared. 
\end{tcolorbox}	
\end{frame}
\begin{frame}

\begin{tcolorbox}[colback=purple!5,colframe=blue!75!black,title=Testing Feature's Impact on Profit]
\begin{itemize}
\item To test whether features like 'Director Facebook Likes','actor\_1\_facebook\_likes' impact movie profits, a different method is adopted. Features are divided into percentiles like 10,20,30 up untill 100.

\item Then its tested if the profit of movies in higher percentile for that feature(say higher than 60 percentile) is higher than that for movies in the lower percentile(less than 60 percent).

\item In this way we are testing if higher values of a particular feature means higher earnings.
\end{itemize}
\end{tcolorbox}	
\end{frame}
\begin{frame}{Results}
\begin{itemize}
\item It was found that Horror movies have significantly higher profits on average than other genres.

\item Movies with aspect ratio of 1.75 and 2.39 were found to have lower profits compared to other aspect ratios.

\item Other features did not have much impact on movie's profits.
\end{itemize}

\end{frame}
\begin{frame}{Summary}
\begin{itemize}
\item The studio is reccomended to make movies in the Horror genre and avoid aspect ratios of 1.75 and 2.39.
\item Multiple hypothesis' being tested implies chances of false positive results. A good way of avoiding this in the future would be to reduce the p-value at which we call the results significant in a way that reduces the False Positive Rate to an acceptable level. 
\end{itemize}
\nocite{fpr} 
\end{frame}
\begin{frame}
\printbibliography

\end{frame}
\end{document}
